\section*{Values}
Money is not only about spreadsheets and numbers. It's about values and behavior. It is important to understand and live by the values which should then shape your finances. In our family we have established 3 values.

\paragraph{God Honoring Stewardship} The first value is making sure we honor God in our finances. The idea of stewardship is to realize our money is not our own but is a gift from God's abundance in our lives. Taking it one step further is not only being good stewards of our finances but focusing first and foremost on honoring God in our stewardship. We honor God in our finances by giving freely and abundantly back to God. \textit{For all things come from you, and of your own have we given you. 1 Chronicles 29:14} We give God first and foremost as our top priority. \textit{Honor the LORD with your wealth and with the first fruits of all your produce. Proverbs 3:9} Finally, in all things we give back to God with gladness in our hearts. \textit{Each one must give as he has decided in his heart, not reluctantly or under compulsion, for God loves a cheerful giver. 2 Corinthians 9:7}

\paragraph{Financial Freedom} The second value is the concept of financial freedom. Not living paycheck to paycheck and if necessary, adjusting our lifestyle to living below our means so we are not bound or controlled by excessive debts. \textit{The rich rules over the poor, and the borrower is the slave to the lender} Also take into consideration the fear (and obedience) to God when financial stability might be tested. \textit{No one can serve two masters, for either he will hate the one and love the other, or he will be devoted to one and despise the other. You cannot serve God and money. Matthew 6:24} Further more, is when debts or taxes are incurred, paying back what is owed and not trying to shirk financial responsibilities. \textit{Pay to all what is owed them: taxes to whom taxes are owed, revenue to whom revenue is owed, respect to whom respect is owed, honor to whom honor is owed. Owe no one anything, except to love each other, for the one who loves another has fulfilled the law. Romans 13;7-8} 

\paragraph{Contentment} The third value is contentment, being content with what we have. \textit{Not that I am speaking of being in need, for I have learned in whatever situation I am to be content. Philippians 4:11} Being too focused on money and greed and the acquisition of more and more wealth and things can lead to temptation and destruction. \textit{But those who desire to be rich fall into temptation, into a snare, into many senseless and harmful desires that plunge people into ruin and destruction. For the love of money is the root of all kinds of evils. It is through this craving that some have wandered away from the faith and pierced themselves with many pangs. 1 Timothy 6:9-10} Finally, remember that live is not just the acquisition of things. \textit{and He said to them, "Take care, and be on your guard against all covetousness, for one's life does not consist in the abundance of his possessions." Luke 12:15}

\section*{Net Worth}
This is the net worth graph going back over 5 years. Each previous year is set in October. Green is total assets, red is total debts and the line is the net worth. Assets minus debts.

\section*{Cash Flow}
This section should be self explainatory. The living, giving, growing, owing is a graphical representation of where the money goes each month.

\section*{Financial Goals}
To track all of our current financial goals. The light blue represents the total amount. The dark blue represents the percentage saved and the red line designates where we should be currently to reach the goal. 

\section*{Assets and Liabilities}
All assets and liabilities by account name and amount. The assets are split between on-budget, which is our primary savings, investments, and checking account and off budget which is our house value (based on the Zillow Zestimate) retirement savings accounts and kids' 529 college savings accounts.

\section*{Financial Ratios}
This section is to evaluate our financial health based on financial ratios not based on other people's financial situation.

\paragraph{Emergency Fund} How much do we have set aside to cover emergencies based as a multiplier of our monthly expenses. This does not include our savings or giving categories. 1x is a good starting point. 3x is the standard. 6x is the gold standard. \\
\textit{Emergency Fund / Total Spending}

\paragraph{Target Net Worth Ratio} This is the percentage of our net worth based on where we should be. This uses the formula from \textit{The Millionaire Next Door} as the basis for calculating this number. House value and mortgage are not factored into this calculation. If below 80\% reducing spending should be a top priority. 80\% - 95\% is a good place to be. Anything above 95\% is the desired goal. \\
\textit{Total Assets / (Age * (Annual Pay / 10))}

\paragraph{House Equity} Measurement of how much equity is in our house against our mortgage. If below 25\% extra monthly payments should be considered to get the total mortgage debt lowered. 25\% - 50\% isn't a bad place to be and over 50\% is great. \\
\textit{(House Value - Mortgage) / Mortgage}

\paragraph{Debt:Asset Ratio} Used to determine how much our total debt is compared to our total assets. If above 36\% steps should be taken to reduce debts as a top priority. 30\% - 36\% is a reasonable place to be with a goal of having total debts being less than 30\% of total assets. \\
\textit{Total Liabilities / Total Assets}

\paragraph{Total Debt:Income Ratio} Used to determine how much of our total annual pay goes towards debt payments. If above 36\% steps should be taken to reduce monthly debt payments. 30\% - 36\% is a reasonable place with a goal of having less than 30\% of monthly expenses going towards debts. \\
\textit{Total Annual Debt Payments / Annual Pay}

\paragraph{Housing:Income Ratio} Used to determine how much of our monthly expenses go towards owning and living in this house. Greater than 36\% is considered bad. 30\% - 36\% is acceptable but under 30\% is a good goal.\\
\textit{(Mortgage + HOA + Maintenance Expenses) / Monthly Net Pay}

\paragraph{Savings Ratio} Used to determine how much of our monthly income is set aside towards savings goals. This number is lower than our Living, Giving, Owing, Growing pie chart because this is based off budget values. The pie chart is based off actual spending. The annual pay value is based off gross pay, not net pay. 10\% - 20\% is a good goal for how much is being saved.\\
\textit{Annual Savings / Annual Pay}

\paragraph{Investments Ratio} This calculates the amount of our total networth that is based on stocks and investments not equities (i.e.; house value). \\
\textit{(Total Assets - Total Equities) / Total Assets}

\paragraph{Retirement Ratio} It is said that you need 25 times your annual salary in order to retire using the 4\% rule. This measures how close we are to that number. It only uses the value of retirement accounts; TSP, IRA in the calculation. \\
\textit{(TSP + IRA) / (Annual Pay * 25)}

\section*{Budget Analysis}

\paragraph{Income} I combined pay, taxes, and insurance to reduce the length of the table. Pay includes Base Pay, BAH, and BAS. Taxes include federal income tax, social security, and medicare. Insurance covers SGLI for self and family and family dental. Roth TSP is self-explainatory. Under actuals, my actual net pay is higher than my salary. This is due to returns on investment and tax returns and other small sources of income.

\paragraph{Expenses} The percentage is based off the percentage of my net pay each category covers. Average spent is the average spent in each category per month over the last 12 months average. Budget category percentage categories are below:
\begin{itemize}
        \item Housing. 30\%
        \item Utilities. 10\%
        \item Transportation. 10\%
        \item Food. 15\%
        \item Entertainment 5\%
        \item Spending Accounts 10\%
        \item Giving 10\%
        \item Saving 10\%
\end{itemize}
